\documentclass{article}
\usepackage[german]{babel}
\usepackage{float}
\usepackage{fourier}
\usepackage[utf8]{inputenc}
\usepackage[T1]{fontenc}
\usepackage{amsfonts,amsthm, amsmath}
\usepackage{listings}
% The following is needed in order to make the code compatible
% with both latex/dvips and pdflatex.
\ifx\pdftexversion\undefined
\usepackage[dvips]{graphicx}
\else
\usepackage[pdftex]{graphicx}
\DeclareGraphicsRule{*}{mps}{*}{}
\fi

\setlength\parindent{0pt}
\lstset{language=Erlang}

\begin{document}

\textbf{Team:} TEAM 01, Falco Winkler (FW), Daniel Schruhl (DS)\\
\\
\textbf{Aufgabenteilung:}
\begin{itemize}
    \item <Aufgabe> (DS, FW)
\end{itemize}

\textbf{Quellenangaben:}\\
\\
\textbf{Bearbeitungszeitraum:}
\begin{itemize}
	\item 13.04.2017 3h (FW)
\end{itemize}

\textbf{Aktueller Stand:}
\begin{itemize}
	\item Projektordner angelegt ;)
\end{itemize}

\textbf{Änderung des Entwurfs:}
\begin{itemize}
    \item Änderungen hier
\end{itemize}

\newpage

\section{Einführung und Ziele}

\subsection{Randbedingungen}

\subsection{Kontextbegrenzung}
Das System soll in Erlang umgesetzt werden. Es muss auf Computern mit Linux Betriebssystem lauffähig sein.

\newpage

\section{Gesamtsystem}

\subsection{Bausteinsicht}


\subsection{Laufzeitsicht}

\newpage

\section{Subsysteme und Komponenten}

\subsection{Starter-Modul}
\subsubsection{Aufgabe und Verantwortung}

Der Starter steht zwischen Koordinator und ggT Prozess. Er übernimmt das starten mehrerer ggT-Prozesse.
Hierfür benötigt er Initialisierungsdaten für die ggT Prozesse, diese werden für ihn durch das Koordinator - Modul
bereitgestellt.


\subsubsection{Schnittstelle}
{steeringval,ArbeitsZeit,TermZeit,Quota,GGTProzessnummer}
Dies ist eine Schnittstelle für den Koordinator. Nach der Anfrage getsteeringval des Starters sendet der Koordinator alle
benötigten Informationen für den Start der Prozesse. Ansprechen dieser Schnitstelle bewirkt das starten von <GGTProzessnummer>
GGT-Prozessen mit gegebenen Werten.
Hierfür zählt eine Rekursive Funktion von GGTProzessnummer bis 0, und startet in jedem Aufruf einen GGT Prozess,
dessen namen sich nach Vorgabe aus den config - Parametern und der Prozessnummer zusammensetzt.
Beim Starten eines GGT Prozesses werden ihm alle benötigten Daten inkl. Starternummer und Adressen von Namensservice und
Koordinator übergeben. Das ringförmige Verketten der Prozesse übernimmt der Koordinator.

\subsubsection{Entwurfsentscheidungen}
Absolute Abstimmungsquote und die Nummer des zu startenden Prozesses werden im Koordinator berechnet und nur übergeben.
Der Starter hat nur die Aufgabe, die Parameter an die gesetzte Anzahl von Clients weiterzureichen.


\subsubsection{Konfigurationsparameter}

\newpage

\subsection{Koordinator-Modul}
\subsubsection{Aufgabe und Verantwortung}

Der Koordinator verwaltet alle ggT Prozesse. Er kommuniziert mit Starter-Prozessen um diesen die benötigten
Werte zum Starten der ggT-Prozesse zu übergeben. Alle ggT-Prozesse müssen sich außerdem bei ihm anmelden,
und er übernimmt die Anordnung dieser in einem Ring, darüber hinaus die Terminierung des gesamten Systems auf Befehl eines
ggt-Prozesses.

Im Koordinator - Modul werden die drei Zustände durch drei receive - Schleifen realisiert. Alle Referenzen auf GGT
Prozesse werden in einer Erlang - Liste persistiert. Meldet sich ein GGT - Prozess beim Koordinator,
wird er entweder am Anfang oder am Ende dieser Liste angefügt. So entsteht die entsprechende Zufälligkeit.

Im Zustandsübergang zu "bereit" wird durch diese Liste iteriert, und die Knoten bekommen ihre entsprechenden Nachbarn
zugewiesen.


\subsubsection{Schnittstelle}

\subsubsection{Entwurfsentscheidungen}

\subsubsection{Konfigurationsparameter}

\end{document}