\documentclass{article}
\usepackage[german]{babel}
\usepackage{float}
\usepackage{fourier}
\usepackage[utf8]{inputenc}
\usepackage[T1]{fontenc}
\usepackage{amsfonts,amsthm, amsmath}
\usepackage{listings}
% The following is needed in order to make the code compatible
% with both latex/dvips and pdflatex.
\ifx\pdftexversion\undefined
\usepackage[dvips]{graphicx}
\else
\usepackage[pdftex]{graphicx}
\DeclareGraphicsRule{*}{mps}{*}{}
\fi

\setlength\parindent{0pt}
\lstset{language=Erlang}

\begin{document}

\textbf{Team:} TEAM 01, Falco Winkler (FW), Daniel Schruhl (DS)\\
\\
\textbf{Aufgabenteilung:}
\begin{itemize}
    \item <Aufgabe> (DS, FW)
    \item Starter (DS)
\end{itemize}

\textbf{Quellenangaben:}\\
\\
\textbf{Bearbeitungszeitraum:}
\begin{itemize}
	\item 13.04.2017 3h (FW)
	\item 24.04.2017 (DS)
\end{itemize}

\textbf{Aktueller Stand:}
\begin{itemize}
	\item Projektordner angelegt ;)
\end{itemize}

\textbf{Änderung des Entwurfs:}
\begin{itemize}
    \item Änderungen hier
\end{itemize}

\newpage

\section{Einführung und Ziele}
Mit dem Satz von Euklid ist es möglich, den größten gemeinsamen Teiler (ggT) zweier positiver ganzer Zahlen
($x,y \in \mathbb{Z}^{*}_{+}$) zu bestimmen (Gleichung \ref{equ:ggt-simple}).

Dabei wird der größte gemeinsame Teiler von $x$ und $y$ auf den größten gemeinsamen Teiler von $y$ und $\bmod{(x,y)}$
zurückgeführt.

\begin{equation}
\forall x,y \in \mathbb{Z}^{*}_{+}: ggT(x,y) = ggT(y,\bmod{(x,y)})
\label{equ:ggt-simple}
\end{equation}

Das erlaubt eine rekursive Berechnung des größten gemeinsamen Teilers.

Das Produkt soll diesen Algorithmus verteilt ausführen, verwalten und koordinieren, um den größten gemeinsamen Teiler zu
berechnen.

\subsection{Randbedingungen}
Um den ggT verteilt mit dem Algorithmus berechnen zu können, muss der Algorithmus angepasst werden
(Gleichung \ref{equ:ggt-distributed}). Das ermöglicht ein Terminieren in jedem ggT-Prozess mit dem ggT und nicht mit 0,
da bei $ggT(x,x)$ terminiert wird.

\begin{multline}
\forall x,y \in \mathbb{Z}^{*}_{+}: ggT(x,y) = ggT(y,\bmod{^{*}(x,y)})\\
\bmod{^{*}(x,y)} := \bmod{(x-1, y)} + 1
\label{equ:ggt-distributed}
\end{multline}

\subsection{Kontextbegrenzung}
Das System soll in Erlang umgesetzt werden. Es muss auf Computern mit Linux Betriebssystem lauffähig sein.

\newpage

\section{Gesamtsystem}

\subsection{Bausteinsicht}


\subsection{Laufzeitsicht}

\newpage

\section{Subsysteme und Komponenten}

\subsection{Starter-Modul}
\subsubsection{Aufgabe und Verantwortung}
Der Starter steht zwischen Koordinator und ggT-Prozess. Er startet mehrere ggT-Prozesse mit den Initialisierungsdaten,
die vom Koordinator zur Verfügung gestellt werden (asynchron) und seinen konfigurierbaren Parametern.

\subsubsection{Schnittstelle}
\begin{lstlisting}
/* Nachricht zum Starten von ggT-Prozessen */
{steeringval,ArbeitsZeit,TermZeit,Quota,GGTProzessnummer}

/* Startet einen Starter Prozess */
start(StarterID): Integer -> PID
\end{lstlisting}
\{\textbf{steeringval},ArbeitsZeit,TermZeit,Quota,GGTProzessnummer\}:\\
Einkommende Nachricht zum Starten von ggT
Prozessen. Die ArbeitsZeit ist die simulierte Verzögerungszeit zur Berechnung in Sekunden, die TermZeit ist die
Wartezeit in Sekunden, bis eine Wahl für eine Terminierung initiiert wird, Quota ist die konkrete Anzahl an
benötigten Zustimmungen zu einer Terminierungsabstimmung und GGTProzessnummer ist die Anzahl der zu startenden
ggT-Prozesse.\\

\textbf{start(}StarterID\textbf{)}:\\
Startet den Starter mit der gegebenen eindeutigen Nummer.\\

\subsubsection{Entwurfsentscheidungen}
Der Starter ruft in seiner Initalisierungsphase an der Schnittstelle vom Koordinator seine benötigten Parameter zum
Starten von ggT-Prozessen asynchron ab. Zusätzlich werden noch die konfigurierbaren Parameter geladen. Mit diesen Daten
werden dann die ggT-Prozesse gestartet. Die Anzahl der zu startenden ggT-Prozesse wird durch die GGTProzessnummer in der
Schnittstelle definiert.

\subsubsection{Konfigurationsparameter}
\begin{itemize}
    \item Praktikumsgruppe
    \item Teamnummer
    \item Nameservicenode definiert den Node des Nameservices
    \item Nameservicename definiert den Namen des registrierten Nameservices auf dem Nameservicenode
    \item Koordinatorname definiert den Namen des Koordinators
\end{itemize}

\newpage

\subsection{Koordinator-Modul}
\subsubsection{Aufgabe und Verantwortung}

Der Koordinator verwaltet alle ggT Prozesse. Er kommuniziert mit Starter-Prozessen um diesen die benötigten
Werte zum Starten der ggT-Prozesse zu übergeben. Alle ggT-Prozesse müssen sich außerdem bei ihm anmelden,
und er übernimmt die Anordnung dieser in einem Ring, darüber hinaus die Terminierung des gesamten Systems auf Befehl eines
ggt-Prozesses.

Im Koordinator - Modul werden die drei Zustände durch drei receive - Schleifen realisiert. Alle Referenzen auf GGT
Prozesse werden in einer Erlang - Liste persistiert. Meldet sich ein GGT - Prozess beim Koordinator,
wird er am Ende dieser Liste angefügt.

Im Zustandsübergang zu "bereit" wird die Liste der GGT-Prozessnummern gemischt, iteriert,
und die Knoten bekommen ihre entsprechenden Nachbarn zugewiesen.


\subsubsection{Schnittstelle}

\subsubsection{Entwurfsentscheidungen}

\subsubsection{Konfigurationsparameter}

\end{document}